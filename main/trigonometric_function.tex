\section{三角函数}\label{zhang_trigonometric_function}
\subsection{三角恒等式}
\begin{align}
    \sin(A+B)= \sin A\cos B+\cos A\sin B\\
    \sin(A-B)= \sin A\cos B-\cos A\sin B\\
    \cos(A+B)= \cos A\cos B-\sin A\sin B\\
    \cos(A-B)= \cos A\cos B+\sin A\sin B
\end{align}
\subsubsection{和差化积}
\begin{align}
    \sin(\alpha)+\sin(\beta)&=2\sin\left(\frac{\alpha+\beta}{2}\right)\cos\left(\frac{\alpha-\beta}{2}\right)\\
    \sin(\alpha)-\sin(\beta)&=2\cos \left(\frac{\alpha+\beta}{2}\right)\sin\left(\frac{\alpha-\beta}{2}\right)\\
    \cos(\alpha)+\cos(\beta)&=2\cos \left(\frac{\alpha+\beta}{2}\right)\cos\left(\frac{\alpha-\beta}{2}\right)\\
    \cos(\alpha)-\cos(\beta)&=-2\sin \left(\frac{\alpha+\beta}{2}\right)\sin\left(\frac{\alpha-\beta}{2}\right)
\end{align}
\subsubsection{积化和差}
\begin{align}
    \cos(A)\sin(B)&=\frac{1}{2}\left[\sin(A+B)-\sin(A-B)\right]\\
    \sin(A)\cos(B)&=\frac{1}{2}\left[\sin(A+B)+\sin(A-B)\right]\\
    \sin(A)\sin(B)&=\frac{1}{2}\left[\cos(A-B)-\cos(A+B)\right]\\
    \cos(A)\cos(B)&=\frac{1}{2}\left[\cos(A+B)+\cos(A-B)\right]
\end{align}
\subsubsection{倍角公式}
\begin{displaymath}
    \centering
    \begin{split}
        \sin(2x)=2\sin x\cos x\\
        \cos(2x)=\cos^2x-\sin^2x=1-2\sin^2 x\\
        \tan(2x)=\frac{2\tan x}{1-\tan^2x}
    \end{split}
\end{displaymath}
\subsubsection{反三角函数}
\begin{align}
    \arcsin x+\arccos x =\frac{\pi}{2}\\
    \arctan x+\operatorname{arccot}{x} =\frac{\pi}{2}
\end{align}
\subsubsection{三角函数其他等式}
\begin{align}
    \sin^2 x+\cos^2 =1\\
    1+\tan^2 x = \sec^2\\
    1+\cot^2 x = \csc^2
\end{align}
\subsection{双曲函数}
\subsubsection{定义}
$$\sinh x = \frac{e^x-e^{-x}}{2}\qquad \cosh x = \frac{e^x+e^{-x}}{2}$$
$$\tanh x = \frac{e^x-e^{-x}}{e^x+e^{-x}}\qquad \coth x = \frac{e^x+e^{-x}}{e^x-e^{-x}}$$
\subsubsection{反双曲函数}
\begin{align}
    \operatorname{arsinh}{x}&=\ln(x+\sqrt{x^2+1})\\
    \operatorname{arcosh}{x}&=\ln(x+\sqrt{x^2-1})\\
    \operatorname{artanh}{x}&=\frac{1}{2}\ln(\frac{1+x}{1-x})
\end{align}

\subsubsection{恒等式}
\begin{align}
&\sinh (2x) = 2\sinh x\cosh x \label{eq:hyperbolic_functions_1} \\
&\cosh^2x-\sinh^2x = 1 \label{eq:hyperbolic_functions_2} \\
&\cosh^2x+\sinh^2x = cosh (2x) \label{eq:hyperbolic_functions_3} \\
&\cosh x = 1+2\sinh^2\frac{x}{2} \label{eq:hyperbolic_functions_4}
\end{align}
