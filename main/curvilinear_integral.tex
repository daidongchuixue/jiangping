\begin{center}\section{曲线及曲面积分}\label{chapter_curvilinear_integral}\end{center}
\subsection{曲线积分定义}
\subsubsection{弧长曲线积分}
$x\circ y$面上曲线L(分段光滑),$f(x,y)$在L上有界,对L进行分割,取积,求和,取极限,如果$\lim\limits_{\lambda\to 0}\sum\limits_{i=0}^{n}f(\xi_i,\eta_i)\Delta S_i$存在,称此极限为$f(x,y)$在L上对弧长的曲线积分,记作
$$\int_{L}f(x,y)ds\triangleq \lim\limits_{\lambda\to 0}\sum\limits_{i=1}^{n}f(\xi_i,\eta_i)\Delta S_i$$
\subsubsection{封闭曲线积分}
如果L是封闭曲线,记作\\
$$\oint_Lf(x,y)ds$$
\subsection{性质}
$$\int_L[\alpha f(x,y)+\beta g(x,y)]ds=\alpha\int_L f(x,y)ds+\beta\int_L g(x,y)ds$$
$$L=L_1+L_2\qquad \int_L f(x,y)ds=\int_{L1} f(x,y)ds+\int_{L2} f(x,y)ds$$
$$f(x,y)\leqslant g(x,y)\qquad \int_L f(x,y)ds\leqslant \int_L g(x,y)ds$$
$$\left|\int_L f(x,y)ds\right|\leqslant \int_L|f(x,y)|ds$$
$$f(x,y)\mbox{在}L\mbox{上连续,则至少窜在一点}(\xi,\eta)\in L\mbox{使}\int_L f(x,y)ds=f(\xi,\eta)l$$
\subsection{弧微分计算}
$f(x,y)$在L上连续,L的参数方程$\begin{cases}
	x=x(t)\\
	y=y(t)
\end{cases}$\quad ($\alpha\leqslant t\leqslant \beta$)\\
若$\frac{dx}{dt},\frac{dy}{dt}$在$[\alpha,\beta]$上连续,且$\left(\frac{dx}{dt}\right)^2+\left(\frac{dy}{dt}\right)^2\neq 0$则\\
$$\int_{L}f(x,y)ds=\int_{\alpha}^{\beta}f(x(t),y(t))\sqrt{[x'(t)]^2+[y'(t)]^2}dt$$
$$L\mbox{极坐标参数方程}\begin{cases}
	x=\rho(\theta)\cos\theta\\
	y=\rho(\theta)\sin\theta
\end{cases}
ds=\sqrt{(\rho d\theta)^2+(d\rho)^2}=\sqrt{\rho^2+\left(\frac{d\rho}{d\theta}\right)^2}d\theta$$
\subsection{对坐标曲线积分}
\subsubsection{定义}
L是$x\circ y$面上的有向光滑曲线,$P(x,y),Q(x,y)$在$L$有界,若$\lim\limits_{\lambda\to 0}\sum\limits_{i=1}^{n}P(\xi_i,\eta_i)\Delta x_i$存在则称此处极限为$P(x,y)$在L上对坐标x的曲线积分,记作
$$\int_{L}P(x,y)dx\triangleq\lim\limits_{\lambda\to 0}\sum\limits_{i=1}^{n}P(\xi_i,\eta_i)\Delta x_i$$
同理定义$Q(x,y)$在L上对坐标y的曲线积分,记作
$$\int_{L}Q(x,y)dy\triangleq\lim\limits_{\lambda\to 0}\sum\limits_{i=1}^{n}Q(\xi_i,\eta_i)\Delta y_i$$
$$w=\int_{L}P(x,y)dx+Q(x,y)dy$$
对坐标的曲线积分称为第二类曲线积分
$$\overrightarrow{r}=\overrightarrow{OM}\qquad d\overrightarrow{r}=(dx,dy)=dx\overrightarrow{i}+dy\overrightarrow{i}$$
$$\int_LPdx+Qdy=\int_L\overrightarrow{F}\bullet d\overrightarrow{r三}$$
\subsubsection{推广}
\begin{align*}
	\overrightarrow{F}=&P(x,y,z)\overrightarrow{i}+Q(x,y,z)\overrightarrow{j}+R(x,y,z)\overrightarrow{k}\\
	w=&\lim\limits_{\lambda\to 0}\sum\limits_{i=1}^{n}\overrightarrow{F}\bullet d\overrightarrow{r}\\
	=&\lim\limits_{\lambda\to 0}\sum\limits_{i=1}^{n}\overrightarrow{F}(\xi_i,\eta_i,\zeta_i)\bullet d\overrightarrow{M_{i-1}M_i}\\
	=&\lim\limits_{\lambda\to 0}\sum\limits_{i=1}^{n}P(\xi_i,\eta_i,\zeta_i)\Delta x_i+Q(\xi_i,\eta_i,\zeta_i)\Delta y_i+R(\xi_i,\eta_i,\zeta_i)\Delta z_i\\
	=&\int_\tau Pdx+Qdy+Rdz
\end{align*}
\subsubsection{性质}
$$\int_{L}\left[\alpha\overrightarrow{F_1}(x,y)+\beta\overrightarrow{F_2}(x,y)\right]\bullet d\overrightarrow{r}=\alpha\int_{L}\overrightarrow{F_1}\bullet d\overrightarrow{r}+\beta\int_{L}\overrightarrow{F_2}\bullet d\overrightarrow{r}$$
$$L+L_1+L_2\qquad \int_L Pdx+Qdy=\int_{L_1} Pdx+Qdy+\int_{L_2} Pdx+Qdy$$
$$L^-\mbox{为有向曲线}L\mbox{的反向曲线}\qquad \int_{L^-}\overrightarrow{F}d\overrightarrow{r}=-\int_{L}\overrightarrow{F}d\overrightarrow{r}$$
\subsubsection{计算}
$P(x,y),Q(x,y)$在有向光滑曲线$L$上连续,$L$的参数方程为$\begin{cases}
	x=x(t)\\
	y=y(t)
\end{cases}$\\
$ t:\ \alpha\mapsto\beta$起点对应$\alpha$,终点对应$\beta,\  x'(t),y'(t)$在$\alpha$到$\beta$区间连续,且$\left[x'(t)\right]^2+\left[y'(t)\right]^2\neq 0$则
$$\int_{L}P(x,y)dx+Q(x,y)dy=\int\limits_{\alpha}^{\beta}P(x(t),y(t))x'(t)+Q(x(t),y(t))y'(t)dt$$
\subsection{两类曲线积分的关系}
$L\mbox{曲线的参数方程}\begin{cases}
	x=x(t)\\
	y=y(t)
\end{cases}\quad t:\ A \mapsto B$\\
$\left[x'(t)\right]^2+\left[y'(t)\right]^2\neq 0\quad \overrightarrow{\tau}=(x'(t),y'(t))$\mbox{为}$L$\mbox{在}$(x(t),y(t))$\mbox{处的一个切向量,方向与增加方向一致}
$$\overrightarrow{T}\triangleq\frac{\overrightarrow{\tau}}{|\overrightarrow{\tau}|}=\left(\frac{x'(t)}{\sqrt{\left[x'(t)\right]^2+\left[y'(t)\right]^2}},\frac{y'(t)}{\sqrt{\left[x'(t)\right]^2+\left[y'(t)\right]^2}}\right)=(\cos\alpha,\cos\beta)$$
	$\int_{L}P(x,y)dx+Q(x,y)dy$\\
	$=\int\limits_{A}^{B}[P(x(t),y(t))x'(t)+Q(x(t),y(t))y'(t)]dt$\\
	$=\int\limits_{A}^{B}\left[P(x(t),y(t))\frac{x'(t)}{\sqrt{\left[x'(t)\right]^2+\left[y'(t)\right]^2}}+Q(x(t),y(t))\frac{y'(t)}{\sqrt{\left[x'(t)\right]^2+\left[y'(t)\right]^2}}\right]\sqrt{\left[x'(t)\right]^2+\left[y'(t)\right]^2}dt$\\
	$=\int_L[P\cos \alpha+Q\cos\beta]ds$
\subsection{格林公式}
$\mbox{区域的正向}\begin{cases}
	\mbox{单连通区域}\quad\mbox{逆时针方向}\\
	\mbox{复连通区域}\quad\begin{cases}
		\mbox{外边界\quad 逆时针}\\
		\mbox{内边界\quad 顺时针}
	\end{cases}
\end{cases}$\\
	$D$是由分段光滑曲线$L$围成,P(x,y),Q(x,y)在D上有一阶连续偏导,则
	\begin{align}
		\iint\limits_{D}\left(\frac{\partial Q}{\partial x}-\frac{\partial P}{\partial y}\right)dxdy=\oint_{L}Pdx+Qdy \label{Green's_formula_1}
	\end{align}
	\subsection{格林公式求面积}
	\begin{align}
		A=\frac{1}{2}\oint_{L}xdy-ydx \label{Green's_formula_2}
	\end{align}
\subsection{曲面积分}
$z=z(x,y)$
\begin{align}
	ds=\sqrt{1+\left(\frac{\partial z}{\partial x}\right)^2+\left(\frac{\partial z}{\partial y}\right)^2}dxdy
\end{align}
\subsection{坐标曲面积分}
规定了曲面法向量方向的曲面称为:有向曲面,双侧曲面$\begin{cases}
	\mbox{外侧,内侧}\\
	\mbox{上侧,内侧}\\
	\mbox{左侧,右侧}\\
	\mbox{前侧,后侧}
\end{cases}$
	
	
	
	
	
	
	
	
	
	
	
	
	
	
	
	
	
