\begin{center}\section{向量导数}\label{chapter_vector_derivative}\end{center}
$$\overrightarrow{r}=x\overrightarrow{i}+y\overrightarrow{j}+z\overrightarrow{k}=(x,y,z)\quad t\in[\alpha,\beta]\begin{cases}
	x=f_1(t)\\
	y=f_2(t)\\
	z=f_3(t)
\end{cases}$$
$$\overrightarrow{f}(t)=f_1(t)\overrightarrow{i}+f_2(t)\overrightarrow{j}+f_3(t)\overrightarrow{k}=(f_1(t),f_2(t),f_3(t))$$
$$\overrightarrow{r}=\overrightarrow{f}(t)\quad t\in[\alpha,\beta]$$
\subsection{向量值函数}
$$D\in R\quad f:D\rightarrow R^n,\mbox{称}f\mbox{为一元向量值函数,}\overrightarrow{r}=\overrightarrow{f}(t)\quad t\in D $$
当$n=3$的情形\qquad$t\in D\rightarrow M(x,y,z)$\\
$\overrightarrow{F}=\overrightarrow{OM}$\ t的变化,M点轨迹$\Gamma$(为空间曲线),称为$\overrightarrow{r}=\overrightarrow{f}(t)$的图形,$\overrightarrow{r}=\overrightarrow{f}(t)$称为$\Gamma$的方程
\subsection{极限}
$\overrightarrow{F}=\overrightarrow{f}(t)$在某个区新领域有定义$\overrightarrow{r}_0$是一个常向量\\
$$\forall\varepsilon>0.\exists\delta>0,0<|t-t_0|<\delta,|\overrightarrow{f}(t)-\overrightarrow{r}_0|<\varepsilon$$
称$\overrightarrow{r}_0$为$\overrightarrow{r}=\overrightarrow{f}(t)$在$t=t_0$处的极限(向量),记作
$$\lim\limits_{t\to t_0}\overrightarrow{f}(t)=\overrightarrow{r}_0\Leftrightarrow \lim\limits_{t\to t_0}\overrightarrow{f}(t)=\left(\lim\limits_{t\to t_0}\overrightarrow{f}_1(t),\lim\limits_{t\to t_0}\overrightarrow{f}_2(t),\lim\limits_{t\to t_0}\overrightarrow{f}_3(t)\right)$$
\subsection{连续}
$\overrightarrow{r}=\overrightarrow{f}(t)$在$t=t_0$处连续
$$\lim\limits_{t\to t_0}\overrightarrow{f}(t)=\lim\limits_{t\to t_0}\overrightarrow{f}(t_0)$$
$\overrightarrow{r}=\overrightarrow{f}(t)$在$D_1\subset D,$上每一点连续,称$\overrightarrow{r}=\overrightarrow{f}(t)$在$D_1$上连续
\subsection{导数}
$\overrightarrow{r}=\overrightarrow{f}(t)$在$t=t_0$某邻域内有定义,如果
$$\lim\limits_{\Delta t\to 0}\frac{\Delta\overrightarrow{r}}{\Delta t}=\lim\limits_{\Delta t\to 0}\frac{\overrightarrow{f}(t+\Delta t)-\overrightarrow{f}(t)}{\Delta t}$$
称此极限为$\overrightarrow{r}=\overrightarrow{f}(t)$在$t=t_0$处的导向量,记作
$$\overrightarrow{f}'(t_0)\quad\mbox{或}\quad \left.\frac{d\overrightarrow{r}}{dt}\right|_{t=t_0}$$
$$\overrightarrow{f}'(t)=\left(\overrightarrow{f}_1'(t),\overrightarrow{f}'_2(t),\overrightarrow{f}'_3(t) \right)$$
\subsection{向量函数求导法则}
\begin{align}
	\frac{d\overrightarrow{c}}{dt}&=\overrightarrow{0}\quad(\overrightarrow{c}\mbox{为常数})\\
	\frac{d\left[c\overrightarrow{u}(t)\right]}{dt}&=c\frac{\overrightarrow{u}(t)}{dt}\\
	\frac{d\left[\overrightarrow{u}(t)\pm\overrightarrow{v}(t)\right]}{dt}&=\frac{\overrightarrow{u}(t)}{dt}\pm\frac{\overrightarrow{v}(t)}{dt}\\
	\frac{d\left[\varphi(t)\overrightarrow{u}(t)\right]}{dt}&=\phi'(t)\overrightarrow{u}(t)+\varphi(t)\overrightarrow{u}'(t)\\
	\frac{d\left[\overrightarrow{u}(t)\bullet\overrightarrow{v}(t)\right]}{dt}&=\overrightarrow{u}'(t)\bullet\overrightarrow{v}(t)+\overrightarrow{u}(t)\bullet\overrightarrow{v}'(t)\\
	\frac{d\left[\overrightarrow{u}(t)\times\overrightarrow{v}(t)\right]}{dt}&=\overrightarrow{u}'(t)\times\overrightarrow{v}(t)+\overrightarrow{u}(t)\times\overrightarrow{v}'(t)\\
	\frac{d\overrightarrow{u}[\varphi(t)]}{dt}&=\overrightarrow{u}'(t)\varphi'(t)
\end{align}
\subsection{曲线的切线与法平面}
$$\Gamma:\begin{cases}
	x=x(t)\\
	y=y(t)\\
	z=z(t)
\end{cases}t\in[\alpha,\beta],t=t_0\mbox{对应点}M(x_0,y_0,z_0)$$
\begin{align*}
	\mbox{切向量:}&\overrightarrow{\Gamma}=(x'(t_0),y'(t_0),z'(t_0))\\
	\mbox{切线方程:}&\frac{x-x_0}{x'(t_0)}=\frac{y-y_0}{y'(t_0)}=\frac{z-z_0}{z'(t_0)}\\
	\mbox{法平面方程:}&x'(t_0)(x-x_0)+y'(t_0)(y-y_0)+z'(t_0)(z-z_0)
\end{align*}
\subsection{曲面的切线与法平面}
空间曲面$\Sigma,F(x,y,z)=0,M(x_0,y_0,z_0)$是$\Sigma$上的一点\\
$$\Gamma\mbox{参数方程}:\begin{cases}
	x=x(t)\\
	y=y(t)\\
	z=z(t)
\end{cases}t\in[\alpha,\beta]$$
$$\Gamma\mbox{在}\Sigma\mbox{上},F(x(t),y(t),z(t))\equiv 0,t\in[\alpha,\beta]$$
$$F'_xx'(t)+F'_yy(t)+F'_zz(t)=0$$
$t=t_0$\mbox{对应点}M
$$F'_x(x_0,y_0,z_0)x'(t_0)+F'_y(x_0,y_0,z_0)y(t_0)+F'_z(x_0,y_0,z_0)z(t_0)=0$$
$\overrightarrow{T}=(x'(t_0),y'(t_0),z'(t_0))$是$\Gamma$过点M的一个切向量\\
$\overrightarrow{n}=(F'_x(x_0,y_0,z_0),F'_y(x_0,y_0,z_0),F'_z(x_0,y_0,z_0))$
$$\overrightarrow{n}\bullet\overrightarrow{T}=0\mbox{,即}\overrightarrow{n}\perp\overrightarrow{T}\quad\overrightarrow{n}\mbox{与所有切线垂直}\overrightarrow{n}\mbox{为切平面的法向量}$$
\subsection{方向导数}
$P_0(x_0,y_0)$以$P_0(x_0,y_0)$为始的的射线L,单位向量$\overrightarrow{e}_l$\\
$\overrightarrow{e}_l(\cos\alpha,\cos\beta)=\cos\alpha\overrightarrow{i}+\cos\beta\overrightarrow{j}$\\
$P_0(x,y)$是$L$上一点$\begin{cases}
	x=x_0+t\cos\alpha\\
	y=y_0+t\cos\beta
\end{cases}\quad t>0$\\
$P_0(x,y)$沿$L$方向趋于$P_0(x_0,y_0)\quad(t\to 0^+)\quad z=f(x,y)$\\
$$\left.\frac{\partial f}{\partial L}\right|_{(x_0,y_0)}\triangleq\lim\limits_{t\to 0^+}\frac{f(x_0+t\cos\alpha,y_0+t\cos\beta)-f(x_0,y_0)}{t}$$
$z=f(x,y)$在点$(x_0,y_0)$处沿L的方向导数
\begin{align}
	\left.\frac{\partial f}{\partial L}\right|_{(x_0,y_0)}=f'_x(x_0,y_0)\cos\alpha+f'_y(x_0,y_0)\cos\beta\label{Directional_derivative}
\end{align}
\subsection{梯度}
$$\overrightarrow{\bigtriangledown} f(x_0.y_0)=\overrightarrow{grad}f(x_0.y_0)\triangleq\left(f'_x(x_0,y_0),f'_y(x_0,y_0)\right)$$
$$\overrightarrow{\bigtriangledown} f(x_0.y_0,z_0)=\overrightarrow{grad}f(x_0.y_0,z_0)\triangleq\left(f'_x(x_0,y_0,z_0),f'_y(x_0,y_0,z_0),f'_z(x_0,y_0,z_0)\right)$$
$$\frac{\partial f}{\partial L}_{(x_0,y_0)}=\overrightarrow{\bigtriangledown} f(x_0.y_0)\bullet\overrightarrow{e}_L\qquad\overrightarrow{e}_L=(\cos\alpha,\cos\beta)$$
\subsection{极值}
$$\mbox{驻点}\begin{cases}
	f'_x(x,y)=0\\
	f'_y(x,y)=0
\end{cases}$$
$$\mbox{记}\begin{cases}
	A=f''_{xx}(x_0,y_0)\\
	B=f''_{xy}(x_0,y_0)\\
	C=f''_{yy}(y_0,y_0)
\end{cases}\quad
\Delta=AC-B^2\begin{cases}
	\Delta>0,z=f(x,y)\mbox{在}(x_0,y_0)\mbox{点}\begin{cases}
		A>0\quad\mbox{极小值}\\
		A<0\quad\mbox{极大值}
	\end{cases}\\
	\Delta=0,\mbox{未定}\\
	\Delta=0,z=f(x,y)\mbox{在}(x_0,y_0)\mbox{点无极值}
\end{cases}$$
