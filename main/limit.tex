\section{极限}\label{zhang_limit}
\subsection{数列极限}
\subsubsection{数列的定义}
\begin{center}
$Def:\qquad \{x_n\},x_n = f(n),n\in N^+\rightarrow R $
\end{center}
\subsubsection{数列极限的定义}
$Def:\ \{x_n\},\ n\in N^+,\exists  a,\ \forall\varepsilon>0,\exists N,\ n>N\Rightarrow \left|x_n-a\right|<\varepsilon$\\
$\lim\limits_{n \to \infty}{x_n}=a$\\ 
$\mbox{极限存在,为收敛,不存在为发散}$
\subsubsection{极限的唯一性}
\begin{align}
\mbox{数列收敛,极限的唯一性}\label{limit_sequence}
\end{align}
\subsubsection{有界数列}
\begin{center}
    $\mbox{若}\exists M>0,\{M\in\mbox{正数}\}$\\
    $\mbox{使得}\forall n,\quad\left|x_n\right|\leqslant M$\\
    $\mbox{则称数列$\{x_n\}$为有界数列}$
\end{center}
\subsubsection{收敛数列与有界性}
\begin{align}
    \mbox{收敛数列必有界}\label{sequence_bounded_1}\\
    \mbox{单调有界数列必收敛}\label{sequence_bounded_2}
\end{align}
\subsubsection{收敛数列的保号性}
\begin{align}
   &\mbox{$\lim\limits_{n \to\infty}x_n=a$存在,且$a>0$,则$\exists N>0,\{N\in N^+\}$当$n>N$时$\Leftrightarrow x_n>0$}\label{Serial_number_preservation_a}\\
    &\lim\limits_{n\to\infty}x_n=a,\lim\limits_{n\to\infty}b_n=b,a<b,\ \exists N,n>N,a_n<b_n \label{Serial_number_preservation_b}
\end{align}
\subsubsection{收敛数列和子数列
}
$\{x_n\},\lim\limits_{n\to\infty}x_n=a,\ \{x_{n_k}\}\subset\{x_n\}\Rightarrow \lim\limits_{n\to\infty}x_{n_k}=a$
\\证明\ $K=N\ k>K$\\
$n_k>n_K\geqslant N$\\
$\left|x_{n_k}-a\right|<\varepsilon$\\
$\lim\limits_{n\to\infty}x_{n_k}=a$

\subsection{函数极限}
\subsubsection{极限的定义}
$Def: \forall \varepsilon >0\begin{cases}
    \exists X>0\begin{cases}
        \mbox{当}x>X&\mbox{时都有}\ \ \left|f(x)-A\right|<\varepsilon \Leftrightarrow \lim\limits_{x\to +\infty}f(x)=A\\
        \mbox{当}x<-X&\mbox{时都有}\ \ \left|f(x)-A\right|<\varepsilon \Leftrightarrow \lim\limits_{x\to -\infty}f(x)=A\\
        \mbox{当}\left|x\right|>X&\mbox{时都有}\ \ \left|f(x)-A\right|<\varepsilon \Leftrightarrow \lim\limits_{x\to \infty}f(x)=A
    \end{cases}\\
    \exists\delta>0\begin{cases}
        \mbox{当}x_0<x<x_0+\delta,\mbox{时}\ \left|f(x)-A\right|<\varepsilon\Leftrightarrow\lim\limits_{x\to x_0^+}f(x)=A\\
        \mbox{当}x_0-\delta<x<x_0 ,\mbox{时}\ \left|f(x)-A\right|<\varepsilon\Leftrightarrow\lim\limits_{x\to x_0^-}f(x)=A\\
        \mbox{当}0<\left|x-x_0\right|<\delta,\mbox{时}\ \left|f(x)-A\right|<\varepsilon\Leftrightarrow\lim\limits_{x\to x_0}f(x)=A
    \end{cases}
\end{cases}$
\begin{center}
    注意1\\定义中$0<\left|x-x_0\right|$表示$x\neq x_0$讨论$x\rightarrow x_0$,只考虑$x\neq x_0$\\
    注意2\\$\lim\limits_{x\to x_0}f(x)$是否存在与$f(x_0)$是否有定义取什么值无关。\\
\end{center}
    \begin{align}
        \lim\limits_{x\to x_0}f(x)\mbox{存在}\Leftrightarrow \lim\limits_{x\to x_0^+}f(x)=\lim\limits_{x\to x_0^-}f(x)\label{limit_left_right}
    \end{align}
图
\subsubsection{极限的性质}
\centerline{1\ 函数的极限的唯一性}
如果$\lim f(x)$存在必唯一。\\
\centerline{2\ 局部有界性}
$\lim\limits_{x\to x_0}f(x)=A,\exists M>0,\delta >0\mbox{使}0<\left|x-x_0\right|<\delta,\left|f(x)\right|\leqslant M$\\
\centerline{3\ 保号性}
$\lim\limits_{x\to x_0}f(x)=A,\ A>0, \exists \delta >0,\mbox{当},0<\left|x-x_0\right|<\delta \Rightarrow f(x)>0$\\
$f(x)>0, \exists \delta >0,\mbox{当},0<\left|x-x_0\right|<\delta \Rightarrow \lim\limits_{x\to x_0}f(x)=A,\ A>0$\\
\centerline{4\ 保序性}
$f(x)\geqslant g(x),\ \lim f(x)=a,\ \lim g(x) = b,\ \mbox{则}a\geqslant b$\\
\centerline{5\ 函数极限与数列极限的关系}
如果$\lim\limits_{x\to x_0}f(x)$存在,$\{x_n\}$为$f(x)$定义域的任一收敛于$x_0$的数列,则满足$x_n\neq x_0$\\
则$\lim\limits_{n\to \infty}f(x_n)=0=\lim\limits_{x\to x_0}f(x),\ x_n\rightarrow x_0$

\subsection{无穷小与无穷大}
\subsubsection{无穷小定义}
$Def:\mbox{如果}\lim\limits_{x\to x_0}f(x)= 0\mbox{则称}f(x)\mbox{为}x\rightarrow x_0\mbox{时的无穷小}$\\
$Def: \forall \varepsilon >0\begin{cases}\exists X>0\begin{cases}
        \mbox{当}x>X&\mbox{时}\ \left|f(x)-0\right|<\varepsilon \Leftrightarrow \lim\limits_{x\to +\infty}f(x)=0\\
        \mbox{当}x<-X&\mbox{时}\ \left|f(x)-0\right|<\varepsilon \Leftrightarrow \lim\limits_{x\to -\infty}f(x)=0\\
        \mbox{当}\left|x\right|>X&\mbox{时}\ \left|f(x)-0\right|<\varepsilon \Leftrightarrow \lim\limits_{x\to \infty}f(x)=0
    \end{cases}\\
    \exists\delta>0\begin{cases}
        \mbox{当}x_0<x<x_0+\delta,\mbox{时}\ \left|f(x)-0\right|<\varepsilon\Leftrightarrow\lim\limits_{x\to x_0^+}f(x)=0\\
        \mbox{当}x_0-\delta<x<x_0 ,\mbox{时}\ \left|f(x)-0\right|<\varepsilon\Leftrightarrow\lim\limits_{x\to x_0^-}f(x)=0\\
        \mbox{当}0<\left|x-x_0\right|<\delta,\mbox{时}\ \left|f(x)-0\right|<\varepsilon\Leftrightarrow\lim\limits_{x\to x_0}f(x)=0
    \end{cases}
\end{cases}$
\subsubsection{函数极限与无穷小的关系}
\begin{align}
    \mbox{在自变量的同一变化中。$\alpha$为无穷小。}\lim f(x)=A\Leftrightarrow f(x)=A+\alpha \label{limit_infinitesimal}
\end{align}
\subsubsection{无穷大与无穷小的关系} 
在自变量同一变化过程中
\begin{center}
    \begin{align}
        \mbox{如果$f(x)$为无穷大,则$\frac{1}{f(x)}$为无穷小。}\label{Infinity_infinitesimal}\\ 
        \mbox{如果$f(x)$为无穷小,切$f(x)\neq 0$,则$\frac{1}{f(x)}$为无穷小。}
    \end{align}
\end{center}
\subsubsection{无穷大定义}
$Def:\forall M>0\begin{cases}
    \exists X>0\begin{cases}
        \mbox{当}x>X\begin{cases}
            f(x)>M\Leftrightarrow \lim\limits_{x\to +\infty}f(x)=+\infty\\
            f(x)<-M\Leftrightarrow \lim\limits_{x\to +\infty}f(x)=-\infty\\
            \left|f(x)\right|>M\Leftrightarrow \lim\limits_{x\to +\infty}f(x)=\infty
        \end{cases}\\
        \mbox{当}x<-X\begin{cases}
            f(x)>M\Leftrightarrow \lim\limits_{x\to -\infty}f(x)=+\infty\\
            f(x)<M\Leftrightarrow \lim\limits_{x\to -\infty}f(x)=-\infty\\
            \left|f(x)\right|>M\Leftrightarrow \lim\limits_{x\to -\infty}f(x)=\infty
        \end{cases}\\
        \mbox{当}\left|x\right|>X\begin{cases}
            f(x)>M\Leftrightarrow \lim\limits_{x\to \infty}f(x)=+\infty\\
            f(x)<-M\Leftrightarrow \lim\limits_{x\to \infty}f(x)=-\infty\\
            \left|f(x)\right|>M\Leftrightarrow \lim\limits_{x\to \infty}f(x)=\infty
        \end{cases}
    \end{cases}\\
    \exists\delta>0\begin{cases}
        \mbox{当}x_0-\delta<x<x_0\begin{cases}
            f(x)>M\Leftrightarrow \lim\limits_{x\to x_0^-}f(x)=+\infty\\
            f(x)<-M\Leftrightarrow \lim\limits_{x\to x_0^-}f(x)=-\infty\\
            \left|f(x)\right|>M\Leftrightarrow \lim\limits_{x\to x_0^-}f(x)=\infty
        \end{cases}\\
        \mbox{当}x_0<x<x_0+\delta\begin{cases}
            f(x)>M\Leftrightarrow \lim\limits_{x\to x_0^+}f(x)=+\infty\\
            f(x)<-M\Leftrightarrow \lim\limits_{x\to x_0^+}f(x)=-\infty\\
            \left|f(x)\right|>M\Leftrightarrow \lim\limits_{x\to x_0^+}f(x)=\infty
        \end{cases}\\
        \mbox{当}0<\left|x-x_0\right|<\delta\begin{cases}
            f(x)>M\Leftrightarrow \lim\limits_{x\to x_0}f(x)=+\infty\\
            f(x)<-M\Leftrightarrow \lim\limits_{x\to x_0}f(x)=-\infty\\
            \left|f(x)\right|>M\Leftrightarrow \lim\limits_{x\to x_0}f(x)=\infty
        \end{cases}
    \end{cases}
\end{cases}$
$$\lim\limits_{x\to x_0}f(x)=\infty,\ \mbox{直线}x=x_0\mbox{是}y=f(x)\mbox{垂直渐进线}$$

\subsection{运算}
\subsubsection{有限个无穷小的和仍为无穷小}
\begin{center}
设$\gamma=\alpha+\beta$\\
$\alpha$和$\beta$同为$x\rightarrow x_0$时的无穷小\\
$\forall \varepsilon >0,\ \exists\delta_1>0,\ $当$0<\left|x-x_0\right|<\delta_1$时,有$\left|\alpha\right|<\frac{\varepsilon}{2}$\\
$\forall \varepsilon >0,\ \exists\delta_2>0,\ $当$0<\left|x-x_0\right|<\delta_2$时,有$\left|\beta\right|<\frac{\varepsilon}{2}$\\
$\delta=min\{\delta_1,\delta_2\},$当$0<\left|x-x_0\right|<\delta$时\\
$0<\left|x-x_0\right|<\delta_1,0<\left|x-x_0\right|<\delta_2$同时满足\\
即$\left|\alpha\right|<\frac{\varepsilon}{2},\left|\beta\right|<\frac{\varepsilon}{2}$同时成立\\
$\left|\gamma\right|=\left|\alpha+\beta\right|<\left|\alpha\right|+\left|\beta\right|< \frac{\varepsilon}{2}+\frac{\varepsilon}{2}=\varepsilon$
\end{center}
\subsubsection{有界函数与无穷小的乘积仍为无穷小}
\begin{center}
    设$\alpha$为$x\rightarrow x_0$时的一个无穷小\\
$g(x)$为$x_0$的一个去心邻域$\mathring{U}(x_0,\delta_1)$有界\\
$f(x)=g(x)\alpha$\\
证$f(x)$为$x\rightarrow x_0$时的无穷小\\
因为$g(x)\mbox{在}\mathring{U}(x_0,\delta_1)$有界\\
$\exists M>0,\mbox{当}0<\left|x-x_0\right|<\delta_1$时$\left|g(x)\right|<M$\\
因为$\alpha$是$x\rightarrow x_0$的无穷小\\
$\exists\delta_2>0\mbox{当}0<\left|x-x_0\right|<\delta_2$时$\left|\alpha\right|<\frac{\varepsilon}{M}<\varepsilon$\\
取$\delta=min\{delta_,\delta_2\}$当$0<\left|x-x_0\right|<\delta$时\\
$\left|g(x)\right|\geqslant M,\left|\alpha\right|<\frac{\varepsilon}{M}$同时成立\\
$\left|g(x)\alpha\right|=\left|g(x)\right|\ \left|\alpha\right|<M\frac{\varepsilon}{M}=\varepsilon$
\end{center}
$$\mbox{推论1.\ 常数与无穷小的乘积为无穷小}$$
$$\mbox{推论2.\ 有限个无穷小的乘积为无穷小}$$
\subsubsection{极限的四则运算}
$\lim f(x)=A,\ \lim g(x)=B$
\begin{align}
&\lim{(f(x)\pm g(x))} = \lim f(x)\pm\lim g(x) \label{Extreme Four Operations_1}\\
&\lim{(f(x) g(x))} = \lim f(x)\lim g(x) \label{Extreme Four Operations_2}\\
&\lim \left(\frac{f(x)}{g(x)}\right)= \frac{\lim{f(x)}}{\lim{g(x)}} \label{Extreme Four Operations_3}\\
&\lim\left[Cf(x)\right]=C\lim f(x)\\
&\lim\left[f(x)\right]^n=\left[\lim f(x)\right]^n
\end{align}
\begin{align}
    \lim\limits_{x\to \infty}\frac{a_0x^m+a_1x^{m-1}+\cdots +a_m}{a_0x^n+a_1x^{n-1}+\cdots +a_n}=\begin{cases}
        \frac{a}{b}\qquad &m=n\\
        \infty &m>n\\
        0 &m<n
    \end{cases}
\end{align}
\begin{equation}
    \begin{split}
        \lim\limits_{x\to x_0}g(x)=u_0,\ \lim\limits_{u\to u_0}f(x)=A\\
        \exists \delta_0>0,\ x\in \mathring{U}\left(x_0,\delta_0\right),\ g(x)\neq u_0\\
        \lim\limits_{x\to x_0}f\left[g(x)\right]=\lim\limits_{u\to u_0}f(u)=A
    \end{split}
\end{equation}

\subsubsection{夹逼定理(三明治定理)}
\vspace{-4mm}
\begin{equation}\label{eq:squeeze_theorem}
\begin{split}
&x_n\leqslant z_n\leqslant y_n \qquad \forall n>N_0 \\
&\mbox{若}\lim\limits_{n\to{\infty}}x_n = \lim\limits_{n\to\infty}y_n = a \mbox{则}\lim\limits_{n\to\infty}z_n = a
\end{split}
\end{equation}
\subsubsection{重要极限}
\vspace{-4mm}
\centerline{$x\rightarrow x_0$}
\vspace{-7mm}
\begin{align}
    \lim\limits_{x\to x_0}\sin x=\sin x_0 \label{limit_3_1}\\
    \lim\limits_{x\to x_0}\cos x=\cos x_0 \label{limit_3_2}
\end{align}
\centerline{$x\rightarrow 0$}
\begin{align}
    \lim\limits_{x\to 0}\frac{\sin x}{x}=1 \label{limit_1_1}\\
    \lim\limits_{x\to 0}\cos x=1 \label{limit_1_2}\\
    \lim\limits_{x\to 0}\frac{\tan x}{x}=1 \label{limit_1_3}\\
    \lim\limits_{x\to 0}\frac{1-\cos x}{\frac{1}{2}x^2}=1 \label{limit_1_4}\\
    \lim\limits_{x\to 0}\frac{\arcsin x}{x}=1 \label{limit_1_5}\\
    \lim\limits_{x\to 0}\frac{\arctan x}{x}=1 \label{limit_1_6}\\
    \lim\limits_{x\to 0}\frac{\ln \left(1+x\right)}{x}=1 \label{limit_1_7}\\
    \lim\limits_{x\to 0}\frac{e^x-1}{x}=1 \label{limit_1_8}\\
    \lim\limits_{x\to 0}\frac{\left(1+x\right)^n-1}{nx}=1 \label{limit_1_9}\\
    \lim\limits_{x\to 0}\left(1+x\right)^\frac{1}{x}=e \label{limit_1_10}
\end{align}
\centerline{$x\rightarrow \infty$}
\begin{align}
    \{x_n\}\qquad\lim\limits_{n\to \infty}\left(1+\frac{1}{n}\right)^n=e \label{limit_2_1}\\
    \lim\limits_{x\to \infty}\left(1+\frac{1}{x}\right)^x=e \label{limit_2_2}
\end{align}
\subsubsection{无穷小比较}
$\frac{0}{0}$型未定式\\
$Def:\ \alpha,\beta$是同一极限过程的无穷小。\\
$\left(1\right)$如果$\lim\frac{\beta}{\alpha}=0$则称$\beta$是$\alpha$的高阶无穷小,记作$\beta=\circ \left(\alpha\right)$\\
$\left(2\right)$如果$\lim\frac{\beta}{\alpha}=\infty$则称$\beta$是$\alpha$的底阶无穷小。\\
$\left(3\right)$如果$\lim\frac{\beta}{\alpha}=C$则称$\beta$是$\alpha$的同阶无穷小。\\
$\left(4\right)$如果$\lim\frac{\beta}{\alpha^k}=C,k>0$则称$\beta$是$\alpha$的k阶无穷小。\\
$\left(5\right)$如果$\lim\frac{\beta}{\alpha}=1$则称$\beta$是$\alpha$的等价阶无穷小。
\subsubsection{等价无穷小代换,因子代换}
$\beta\mbox{与}\alpha\mbox{是等价无穷小}\Leftrightarrow\beta=\alpha +\circ \left(\alpha\right)$\\
$\mbox{设}\alpha \sim \alpha',\ \beta \sim \beta',\mbox{且}\lim\frac{\beta'}{\alpha'}\mbox{存在,则}\lim\frac{\beta}{\alpha}=\lim\frac{\beta'}{\alpha'}$ \\
$\lim\alpha f(x)=\lim\alpha'f(x)$\\
$\lim\frac{f(x)}{\alpha} =\lim\frac{f(x)}{\alpha'}$