\section{函数与图像}
\subsection{函数的定义}
设数集$D\in R$的映射
$$f: D\rightarrow R$$
称f为定义在D上的函数,记为
$$y = f(x)\ \{x\in D\}$$
 \subsection{函数的性质}
 \subsubsection{函数的有界性}
 $f:D\rightarrow R\{D\subset R\}$$\begin{cases}
    \mbox{有界}\begin{cases}
        \mbox{有上界}\begin{cases}
            \exists k_1,\ \mbox{使}f(x)\leqslant k_1,\ \forall x\in D  
        \end{cases}\\
        \mbox{有下界}\begin{cases}
            \exists k_1,\ \mbox{使}f(x)\geqslant  k_1,\ \forall x\in D  
        \end{cases}
    \end{cases}\\
    \mbox{无界}\begin{cases}
        \mbox{无上界}\begin{cases}
            \forall K_1,\ \exists x\in D\ \mbox{使},\ f(x)\geqslant  k_1
        \end{cases}\\
        \mbox{无下界}\begin{cases}
            \forall K_1,\ \exists x\in D\ \mbox{使},\ f(x)\leqslant  k_1
        \end{cases}
    \end{cases}
 \end{cases}$
 \subsubsection{函数的单调性}
 单调增加
 $\mbox{若}\{x_1,x_2\in D\}\ x_1<x_2\Rightarrow \begin{cases}
    f(x_1)<f(x_2)  \mbox{称}f(x)\mbox{在D上单调增加}\\
    f(x_1)>f(x_2)  \mbox{称}f(x)\mbox{在D上单调减少}\\
    f(x_1)\leqslant f(x_2)  \mbox{称}f(x)\mbox{在D上单调非降}\\
    f(x_1)\geqslant f(x_2)  \mbox{称}f(x)\mbox{在D上单调非增}
 \end{cases}$
 \subsubsection{函数的奇偶性}
 定义域\\ 
 \bigskip
 $\forall x\in D\qquad f(-x)=\begin{cases}
    f(x)\qquad &\mbox{偶函数}\\
    -f(x) &\mbox{奇函数}\\
 \end{cases}$\\
 \bigskip
 奇偶性运算
\begin{align}
    \mbox{奇函数}\times \mbox{奇函数}=\mbox{偶函数}\\
    \mbox{奇函数}\times \mbox{偶函数}=\mbox{奇函数}\\
    \mbox{偶函数}\times \mbox{偶函数}=\mbox{偶函数}
 \end{align}
 \subsubsection{周期性}
 $Def:\quad f(x+L)=f(x) \{L>0\mbox{常数},\forall x\in D\}\Rightarrow \mbox{$f(x)$为$L$的周期函数}$
 \subsection{函数图像} 
    \begin{tikzpicture}[>=latex]
        \centering
        \begin{scope}
            \draw[->](0,0)--(6,0);
            \draw[->](3,-1)--(3,2);
            \draw[-](0,1)--(6,1)node[below]{a};
            \node at (3,-1)[below]{常函数\ $f(x)=a\{a\in R\}$};
        \end{scope}
        \begin{scope}[xshift=6.5cm]
            \draw[->](0,0)--(6,0);
            \draw[->](3,-1)--(3,2);
            \draw(1,2)--(3,0)--(5,2);
            \node at (3,-1)[below]{$f(x)=\left|x\right|$};
        \end{scope}
        \begin{scope}[yshift=-3.5cm,yscale=.8]
            \draw[->](6,0)--(12,0);
            \draw[->](9,-1)--(9,2);
            \draw[-](9,1)--(12,1);
            \draw[-](6,-1)--(9,-1);
            \node at (3,.5){$f(x)=sgn\ x=\begin{cases}
                 1 &x>0 \\
                 0&x=0 \\
                 -1\ &x<0
            \end{cases}$};
            \node at (6,-1.5){$\left|x\right|  = x\cdot sgnx$};
        \end{scope}
        \begin{scope}[yshift=-6cm,scale=.4]
            \draw[->](0,0)--(4*pi+.1*pi,0);
            \draw[->](2*pi,-1.5)--(2*pi,1.5);
            \draw[domain=0:4*pi,samples=1000] plot(\x,{sin(\x r)});
            \foreach \x/\xtext in {{1/2}/\frac{\pi}{2},{3/2}/\frac{3\pi}{2},1/\pi,2/2\pi}
                {
                    \draw[dashed] (\x*pi+2*pi,{sin(\x*pi r+2*pi)})--(\x*pi+2*pi,0)node[below]{$\xtext$};
                    \draw[dashed] (2*pi-\x*pi,{-sin(2*pi r-\x*pi r)})--(2*pi-\x*pi,0)node[below]{$-\xtext$};
                }
            \node at(2*pi,-1.5)[below]{$\sin x$};
        \end{scope}
        \begin{scope}[yshift=-6cm,xshift=6cm,scale=.4]
            \draw[->](0,0)--(4*pi+.1*pi,0);
            \draw[->](2*pi,-1.5)--(2*pi,1.5);
            \draw[domain=0:4*pi,samples=1000] plot(\x,{cos(\x r)});
            \foreach \x/\xtext in {{1/2}/\frac{\pi}{2},{3/2}/\frac{3\pi}{2},1/\pi,2/2\pi}
                {
                    \draw[dashed] (\x*pi+2*pi,{cos(\x*pi r+2*pi)})--(\x*pi+2*pi,0)node[below]{$\xtext$};
                    \draw[dashed] (2*pi-\x*pi,{cos(2*pi r-\x*pi r)})--(2*pi-\x*pi,0)node[below]{$-\xtext$};
                }
            \node at(2*pi,-1.5)[below]{$\cos x$};
        \end{scope}
        \begin{scope}[yshift=-8cm,scale=.4]
            \draw[->](0,0)--(4*pi+.1*pi,0);
            \draw[->](2*pi,-2)--(2*pi,2);
            \foreach \x in {0,1,2,3}{
                \draw[dashed](\x*pi+pi/2,-2)--(\x*pi+pi/2,2);
            }
            \draw[domain=0:pi/2-.15*pi,samples=1000] plot(\x,{tan(\x r)});
            \draw[domain=pi/2+.15*pi:3*pi/2-.15*pi,samples=1000] plot(\x,{tan(\x r)});
            \draw[domain=3*pi/2+.15*pi:5*pi/2-.15*pi,samples=1000] plot(\x,{tan(\x r)});
            \draw[domain=5*pi/2+.15*pi:7*pi/2-.15*pi,samples=1000] plot(\x,{tan(\x r)});
            \draw[domain=7*pi/2+.15*pi:8*pi/2-.15*pi,samples=1000] plot(\x,{tan(\x r)});
            \node at(2*pi,-2)[below]{$\tan x$};
        \end{scope}
        \begin{scope}[yshift=-8cm,xshift=6cm,scale=.4]
            \draw[->](0,0)--(4*pi+.1*pi,0);
            \draw[->](2*pi,-2)--(2*pi,2);
            \foreach \x in {0,1,2,3,4}{
                \draw[dashed](\x*pi,-2)--(\x*pi,2);
            }
            \draw[domain=0+.15*pi:pi-.15*pi,samples=1000] plot(\x,{cot(\x r)});
            \draw[domain=pi+.15*pi:2*pi-.15*pi,samples=1000] plot(\x,{cot(\x r)});
            \draw[domain=2*pi+.15*pi:3*pi-.15*pi,samples=1000] plot(\x,{cot(\x r)});
            \draw[domain=3*pi+.15*pi:4*pi-.15*pi,samples=1000] plot(\x,{cot(\x r)});
            \node at(2*pi,-2)[below]{$\cot x$};
        \end{scope}
        \begin{scope}[yshift=-11cm,scale=.4]
            \draw[->](0,0)--(4*pi+.1*pi,0);
            \draw[->](2*pi,-3)--(2*pi,3);
            \foreach \x in {0,1,2,3}{
                \draw[dashed](\x*pi+pi/2,-3)--(\x*pi+pi/2,3);
            }
            \draw[domain=0:pi/2-.1*pi,samples=1000] plot(\x,{sec(\x r)});
            \draw[domain=pi/2+.1*pi:3*pi/2-.1*pi,samples=1000] plot(\x,{sec(\x r)});
            \draw[domain=3*pi/2+.1*pi:5*pi/2-.1*pi,samples=1000] plot(\x,{sec(\x r)});
            \draw[domain=5*pi/2+.1*pi:7*pi/2-.1*pi,samples=1000] plot(\x,{sec(\x r)});
            \draw[domain=7*pi/2+.1*pi:8*pi/2,samples=1000] plot(\x,{sec(\x r)});
            \node at(2*pi,-3)[below]{$\sec x$};
        \end{scope}
        \begin{scope}[yshift=-11cm,xshift=6cm,variable=\t,scale=.4]
            \draw[->](0,0)--(4*pi+.1*pi,0);
            \draw[->](2*pi,-3)--(2*pi,3);
            \foreach \x in {0,1,2,3}{
                \draw[dashed](\x*pi,-3)--(\x*pi,3);
            }
            \draw[domain=0+.1*pi:pi-.1*pi,samples=1000] plot(\t,{1/sin(\t r)});
            \draw[domain=pi+.1*pi:2*pi-.1*pi,samples=1000] plot(\t,{1/sin(\t r)});
            \draw[domain=2*pi+.1*pi:3*pi-.1*pi,samples=1000] plot(\t,{1/sin(\t r)});
            \draw[domain=3*pi+.1*pi:4*pi-.1*pi,samples=1000] plot(\t,{1/sin(\t r)});
            \node at(2*pi,-3)[below]{$\csc x$};
        \end{scope}
    \end{tikzpicture}
