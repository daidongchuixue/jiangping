\section{函数与图像}
\subsection{函数的定义}
设数集$D\in R$的映射
$$f: D\rightarrow R$$
称f为定义在D上的函数,记为
$$y = f(x)\ \{x\in D\}$$
\subsection{函数的类型}
%\begin{figure}[htp]
%    $\mbox{常数}$
%    \centering
%    \begin{tikzpicture}[>=latex]
       % \begin{axis}[
         %   xlabel=$x$,ylabel=$y$,
         %   xmin=0,xmax=10,
          %  title={常函数}]
        % \end{axis}
    %\end{tikzpicture}
    %\caption{asd}\label{dsa}
%\end{figure}
\begin{figure}[htp]
   % \centering
    \begin{tikzpicture}[>=latex]
       \draw[->](-3,0)--(3,0);
       \draw[->](0,-1)--(0,2);
       \draw[-](-3,1)--(3,1)node[below]{a};
       \node at (0,-1)[below]{常函数\ $f(x)=a\{a\in R\}$};
    \end{tikzpicture}
    %\caption{常函数\ $f(x)=a$}%\label{dsa}
        % \centering
        \begin{tikzpicture}[>=latex]
            \draw[->](-3,0)--(3,0);
            \draw[->](0,-1)--(0,2);
            \draw(-2,2)--(0,0)--(2,2);
            \node at (0,-1)[below]{$f(x)=\left|x\right|$};
         \end{tikzpicture}
         %\caption{常函数\ $f(x)=a$}%\label{dsa}
\end{figure}
\begin{figure}[htp]
     \centering
     \begin{tikzpicture}[>=latex]
        \draw[->](-3,0)--(3,0);
        \draw[->](0,-1)--(0,2);
        \draw[-](0,1)--(3,1);
        \draw[-](-3,-1)--(0,-1);
        \node at (-6,.5){$f(x)=sgn\ x=\begin{cases}
            1\qquad x>0 \\
            0\qquad x=0 \\
            -1\qquad x<0
        \end{cases}$};
     \end{tikzpicture}
 \end{figure}
 $$\left|x\right| = x\ sgnx$$
 \subsection{函数的性质}
 \subsubsection{函数的有界性}
 $f:D\rightarrow R\{D\subset R\}$$\begin{cases}
    \mbox{有界}\begin{cases}
        \mbox{有上界}\begin{cases}
            \exists k_1,\ \mbox{使}f(x)\leqslant k_1,\ \forall x\in D  
        \end{cases}\\
        \mbox{有下界}\begin{cases}
            \exists k_1,\ \mbox{使}f(x)\geqslant  k_1,\ \forall x\in D  
        \end{cases}
    \end{cases}\\
    \mbox{无界}\begin{cases}
        \mbox{无上界}\begin{cases}
            \forall K_1,\ \exists x\in D\ \mbox{使},\ f(x)\geqslant  k_1
        \end{cases}\\
        \mbox{无下界}\begin{cases}
            \forall K_1,\ \exists x\in D\ \mbox{使},\ f(x)\leqslant  k_1
        \end{cases}
    \end{cases}
 \end{cases}$
 \subsubsection{函数的单调性}
 单调增加
 $\mbox{若}\{x_1,x_2\in D\}\ x_1<x_2\Rightarrow \begin{cases}
    f(x_1)<f(x_2)  \mbox{称}f(x)\mbox{在D上单调增加}\\
    f(x_1)>f(x_2)  \mbox{称}f(x)\mbox{在D上单调减少}\\
    f(x_1)\leqslant f(x_2)  \mbox{称}f(x)\mbox{在D上单调非降}\\
    f(x_1)\geqslant f(x_2)  \mbox{称}f(x)\mbox{在D上单调非增}
 \end{cases}$
 \subsubsection{函数的奇偶性}
 定义域\\ 
 \bigskip
 $\forall x\in D\qquad f(-x)=\begin{cases}
    f(x)\qquad &\mbox{偶函数}\\
    -f(x) &\mbox{奇函数}\\
 \end{cases}$\\
 \bigskip
 奇偶性运算
\begin{align}
    \mbox{奇函数}\times \mbox{奇函数}=\mbox{偶函数}\\
    \mbox{奇函数}\times \mbox{偶函数}=\mbox{奇函数}\\
    \mbox{偶函数}\times \mbox{偶函数}=\mbox{偶函数}
 \end{align}
 \subsubsection{周期性}
 $Def:\quad f(x+L)=f(x) \{L>0\mbox{常数},\forall x\in D\}\Rightarrow \mbox{$f(x)$为$L$的周期函数}$