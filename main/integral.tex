\section{不定积分}
\subsection{概念}
\subsubsection{原函数}
$$\forall x\in I,\ F'(x)=f(x),\quad F(x)\mbox{为}f(x)\mbox{的一个原函数}$$
\begin{align}
    \mbox{函数}f(x)\mbox{在区间}I\mbox{上连续一定有}F(x),\mbox{使}F'(x)=f(x)
\end{align}
\subsubsection{不定积分}
区间$I$上,$f(x)$带有任意常数的原函数,称为$f(x)$在区间I上的不定积分。\\
记作:
$$\int f(x)dx\quad\begin{cases}
    \int&\mbox{积分符号}\\
    f(x)&\mbox{被积函数}\\
    f(x)dx\ &\mbox{被积表达式}\\
    x &\mbox{积分变量}
\end{cases}$$
$\mbox{如果}F(x)\mbox{是}f(x)\mbox{的一个原函数}$
$$\int f(x)dx=F(x)+C$$
\subsubsection{不定积分性质}
$$\left[\int f(x)dx\right]'=f(x)$$
\subsection{幂数,指数,对数}
\begin{align}
&\int k dx=kx +C\\
&\int x^a \mathrm{d}{x} = \frac{x^{a+1}}{a+1} + C\\
&\int a^x \mathrm{d}{x} = \frac{a^x}{\ln a} + C\\
&\int e^x \mathrm{d}{x} = e^x + C\\
&\int \frac{1}{x} \mathrm{d}{x} = \ln\left|x\right| + C
\end{align}
\subsection{三角函数}
\begin{align}
&\int \frac{1}{1+x^2}=\arctan x + C\\
&\int\sin x \mathrm{d}{x} = -\cos x + C\\
&\int\frac{1}{\sqrt{1-x^2}}\mathrm{d}{x} =\begin{cases}
    \arcsin x + C\\
    -\arccos x +C_1
\end{cases}  \\
&\int\csc x\cot x \mathrm{d}{x} = -\csc x + C\\
&\int\cos x \mathrm{d}{x} = \sin x + C \\
&\int\sec x \tan x\mathrm{d}{x} = \sec x + C \\
&\int\sec^2 x\mathrm{d}{x} = \tan x + C \\
&\int\csc^2 x\mathrm{d}{x} = -\cot x +C \\
&\int\frac{1}{\left|x\right|\sqrt{x^2-1}}\mathrm{d}{x} = \operatorname{arcsec} x + C \\
&\int\frac{1}{1+x^2}\mathrm{d}{x} = \tan x + C\\
&\int\sinh x \mathrm{d}{x} = \cosh x + C\\
&\int\cosh x \mathrm{d}{x} = \sinh x + C
\end{align}
\subsection{积分运算}
