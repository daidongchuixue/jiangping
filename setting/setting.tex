\usepackage{makecell} %单元格内换行

\usepackage{amsmath,mathrsfs,amsfonts} %公式手动编码不能再 $ $$中
\usepackage{yhmath}%AB弧
\usepackage{extpfeil}%等号上面写东西
\usepackage{tikz}
\usepackage{calc}
\usepackage{graphicx}
%\usepackage{pgfplots}

\usepackage{amssymb}
\usepackage{cases}  %方程组每一个都编码
%\usepackage{sectsty} % 引入sectsty包 设置section字体大小这个包有两个警告,和别的包冲突
\usepackage{pifont}
%\sectionfont{\fontsize{11}{12}\selectfont\centering}% 设置section的字体大小
%\subsectionfont{\fontsize{11}{12}\selectfont}% 设置section的字体大小


\usepackage[a5paper,left=5mm,right=4mm,top=14mm,bottom=1mm]{geometry}% 设置页面的环境,a4纸张大小,左右上下边距信息
\usepackage{fancyref}
%\usepackage{graphicx}

\usepackage{indentfirst}% 使用indentfirst宏包

\setlength{\parindent}{0em} % 设置首行缩进距离


\headsep=2mm

\usepackage{fontawesome5}
\usepackage{fancyhdr}%设置页码
\fancypagestyle{mystyle}{%
	\fancyhf{} % 清空页眉和页脚的设置
	%\rhead{\thepage}
	\setcounter{page}{1}
	%\fancyhead[L]{\thepage} % 在页眉左侧显示页码
	\fancyhead[R]{\S\leftmark$\quad|\qquad$\textbf{\thepage}} % 在页眉中间显示当前章的标题
	%\fancyhead[R]{\rightmark} % 在页眉右侧显示当前节的标题
	%\renewcommand{\headrulewidth}{0.4pt} % 设置页眉下方横线的粗细为0.4pt
	%\renewcommand{\footrulewidth}{0pt} % 页脚上方无横线
}
